\chapter*{Introducción}\label{chapter:introduction}
\addcontentsline{toc}{chapter}{Introducción}

\section{Contexto hist\'orico/social donde se desarrolla la investigaci\'on}
\hspace{.1cm}La modelación matemática, física y computacional de fenómenos biológicos se ha convertido en una herramienta esencial para la investigación científica en el siglo XXI. Se utiliza en una amplia variedad de campos, como la medicina, ingeniería y economía. La misma ha tenido un impacto significativo en la forma en que abordamos y entendemos los problemas del mundo real. Ha permitido tanto a científicos como a médicos entender y predecir el comportamiento de enfermedades como el cáncer, lo que ha llevado a avances significativos en el tratamiento y prevención de esta enfermedad.

\hspace{.1cm}La Organización Mundial de la Salud (OMS) proporciona datos sobre el impacto del cáncer a nivel mundial, el cual es una de las principales causas de muerte en todo el mundo, son casi 10 millones de muertes en 2020[3] y aún se desconoce una cura para el mismo. Un artículo publicado en PubMed[4] proporciona una actualización de la carga global de este fen\'omeno utilizando las estimaciones GLOBOCAN 2020 de incidencia y mortalidad producidas por la Agencia Internacional de Investigación sobre el Cáncer. A nivel mundial se estimaron 19.3 millones de nuevos casos (18.1 millones excluyendo el cáncer de piel no melanoma) y casi 10 millones de muertes en 2020 [4].

\hspace{.1cm}Según un estudio publicado en el National Center for Biotechnology Information (NCBI) el cáncer es la segunda causa de muerte en Cuba, representando el 25\% de todas las muertes. En 2015, Cuba reportó 44,454 nuevos casos de cáncer, con una tasa de incidencia bruta de 425.6 casos por cada 100,000 habitantes hombres y 366.7 en mujeres[2]. Desde 1992, Cuba ha tenido un Programa Nacional de cáncer, dirigido por el Ministerio de Salud, que coordina todos los componentes del control del mismo, incluyendo la comunicación, la participación de los individuos, la prevención, el diagnóstico temprano y el tratamiento. También supervisa la gestión de recursos y la política de investigación.[2] Debido a estos motivos se hace necesaria alguna herramienta que contribuya al entendimiento de esta enfermedad.

\section{Antecedentes del problema científico}
\hspace{.1cm}El cáncer es una enfermedad compleja que ha sido estudiada durante siglos. Es una enfermedad genética, causada por los cambios en los genes que controlan el funcionamiento de las células, en especial, como se forman y multiplican. Está fuertemente influenciada por factores genéticos, ambientales y evolutivos. Algunas células del organismo se multiplican sin control y se diseminan a otras partes del cuerpo[1]. La malignidad y alcance de los daños que puede provocar es variable y depende de la velocidad de crecimiento de las células oncológicas, la capacidad de estas últimas de propagarse a otros tejidos y la posibilidad de reaparecer una vez que son removidas quirúrgicamente. En el pasado, los investigadores han desarrollado varios modelos para describir y entender el crecimiento tumoral. Estos modelos han variado desde el uso de ecuaciones diferenciales ordinarias hasta el uso de autómatas celulares. A pesar de estos avances todavía hay aspectos del crecimiento tumoral que no se comprenden completamente, como la interacción entre las células cancerígenas y el sistema circulatorio. Al llevarse a cabo este tipo de investigación se tiene como objetivo aprovechar la mayor cantidad de recursos teóricos a nuestra disposición y unirlos en la práctica con la simulación de procesos biológicos para obtener resultados que favorezcan el avance científico en los campos donde pueda aplicarse.

\section{Actualidad} 
\hspace{.1cm}Los avances recientes en el tratamiento del cáncer han llevado a la introducción de terapias más precisas y menos invasivas. Entre estas terapias innovadoras, la terapia dirigida y la inmunoterapia han surgido como las metodologías más prometedoras. La terapia dirigida interfiere con las moléculas específicas que ayudan al crecimiento y la proliferación de las células cancerosas. Por ejemplo, la quercetina, un flavonoide polifen\'olico, ha demostrado ser efectiva para tratar varios tumores al interferir con muchas vías de señalización[5]. Por otro lado la inmunoterapia busca potenciar la respuesta del sistema inmune del propio cuerpo para combatir enfermedades. Actualmente se están realizando numerosos ensayos clínicos para investigar nuevas estrategias de terapia génica para el cáncer, incluyendo la expresión de genes proapoptoticos y quimiosensibilizadores, también el silenciamiento dirigido de oncogenes[6].

\hspace{.1cm}En los últimos años numerosos investigadores han explotado las facilidades que ofrecen los autómatas celulares para razonar en términos de individuos dado que en este marco la población y el tiempo son variables discretas. La dinámica de un autómata celular depende de un conjunto de estados que toman las células de la población y una función de transición que expresa los cambios que ocurren entre dichos estados[7]. Luego, se pueden representar estos estados de las células como un conjunto de celdas las cuales se agrupan para formar una rejilla, la cual puede ser de una línea, un plano de 2 dimensiones o un espacio n-dimensional. Los autómatas celulares han demostrado ser herramientas útiles en diversas aplicaciones. Pueden utilizarse para simular la dinámica de las células en un tejido biológico o para modelar la propagación de una enfermedad en una población. Gracias a su capacidad para representar interacciones locales y cambios de estado en pasos discretos los autómatas celulares son capaces de capturar la dinámica compleja de estos sistemas biológicos con un alto grado de realismo. Además, se pueden utilizar para simular la respuesta de un sistema biológico a diferentes terapias, como la quimioterapia o la radioterapia.

\section{Antecedentes} 
\hspace{.1cm}El término crecimiento puede ser empleado para referirse a un incremento de tamaño o volumen, como para indicar un aumento en una muestra o poblaci\'on. Se pueden encontrar tres tipos de investigadores cuyos esquemas básicos tienen excelentes reseñas y se consideran que son correctos para el tratamiento de problemas prácticos, estos son: el modelo de crecimiento exponencial de Thomas R. Malthus(1798), la ley de crecimiento de Benjamin Gompertz(1825) y el modelo de crecimiento log\'istico de Pierre F. Verhulst (1838).[7]. En [8] se presenta una comparación de estos modelos y en [9] se exponen relaciones entre los mismos tomando como base de la comparación los datos provenientes de sistemas de cultivo de esferoides tumorales multicelulares (de ahora en adelante MTS, \textit{multicellular tumor spheroids})[7], también en [10] se emplea el modelo de Gomperts para simular el crecimiento de un tumor en el cerebro.

\hspace{.1cm}Existen otras áreas de gran interés que proveen otros enfoques para tratar el crecimiento tumoral, entre estas se encuentra el \'area de modelaci\'on de variables continuas, sobresaliendo la mecánica de medios continuos (de ahora en adelante MMC). En este, el comportamiento de un material se rige por medio de ecuaciones constitutivas que caracterizan las propiedades de dicho material, y leyes de balance entre las que se encuentran la ley de balance de masa y de energía [7]. La evolución del tumor se deriva de las ecuaciones de balance y de principios de conservación suplementados con leyes de difusión con el objetivo de describir la evolución de los nutrientes que el tumor recibe para su desarrollo [11]. Algunos de los trabajos de estudios que se basan en MMC ([12] y [13]) consideran que todas las células del organismo poseen una tensión ideal y si en algún momento esta varía existen mecanismos para devolverla a su estado ideal. Esta hipótesis tiene su base en los mecanismos de homeost\'asis del organismo, que consiste en la capacidad del mismo para mantener una condición interna estable. Por tanto, la aparición de un tumor se interpreta como una falla en los mecanismos que hacen que la célula, y por extensión el tejido, recuperen la tensión ideal. Se pueden encontrar ejemplos en [14].

\hspace{.1cm}Los autómatas celulares han sido utilizados extensamente para simular el crecimiento de un tumor durante la etapa avascular[15,16], la invasión celular[17] y sus interacciones con varios factores del entorno [18]. En el [19] se simula el crecimiento de un tumor y se tienen en cuenta las interacciones de las células cancerígenas, sanas y las pertenecientes al sistema inmunológico. Se pueden obtener comparaciones de las fortalezas y debilidades de varios modelos basados en células, incluidos los autómatas celulares en varios escritos [18].

\section{Novedad científica} 
\hspace{.1cm}Como problema cient\'ifico de este tipo de investigación se pretende alcanzar un entendimiento más profundo de los procesos biológicos a través de un ciclo iterativo de teoría y experimentación. Además, los modelos matemáticos pueden ser empleados para asistir en la concepción y diseño de estrategias terapéuticas, proporcionando una visión más precisa y personalizada del tratamiento de cada paciente.

\hspace{.1cm}Luego, en el caso de este proyecto, como pregunta científica o hip\'otesis se tiene que se concibe un modelo de autómatas celulares que presenta de forma integral las etapas avascular y vascular del desarrollo tumoral, así como los procesos de invasión, migración y metástasis. Se emplea una red de mundo pequeño[20](agregar algo mas). Los parámetros y configuraciones pueden ser cargados desde archivos externos, ofreciendo una gran flexibilidad en la adaptación de la simulación a las necesidades específicas de cada caso. Se tiene como objetivo simular el crecimiento de tumores en órganos pequeños e involucra la carga y utilización de parámetros para la simulación. El gráfico de las células y sus conexiones se visualiza y analiza, junto con la visualización del tamaño del tumor a lo largo de la simulación. Se implementan algoritmos eficientes para procesar grandes cantidades de células y sus conexiones.

\hspace{.1cm}La técnica de Marching Cubes[21] se utiliza para la renderizaci\'on en 3D, porporcionando una visualizaci\'on detallada y precisa de los tumores. La visualizaci\'on resultante puede proporcionar una comprension valiosa de cómo se desarrolla y propaga el cancer, lo que puede ser esencial para el desarrollo de terapias y tratamientos efectivos. Al visualizar el crecimiento del tumor en tres dimensiones, los me\'dicos y cient\'ificos pueden obtener una mejor comprensi\'on de la evoluci\'on del tumor y cómo puede afectar a los tejidos circundantes. Esta informaci\'on puede ser crucial para el desarrollo de terapias y tratamientos.

\section{Objetivos y contribuciones} 
\hspace{.1cm}El objetivo general de este trabajo es implementar una herramienta a trav\'es de un aut\'omata celular que simule el surgimiento, desarrollo y met\'astasis de un tumor originado en el tejido epitelial de los \'organos, as\'i como su interacción con el sistema inmune y la influencia de diversos factores internos y externos en su evoluci\'on. Se esperan obtener resultados similares a los obtenidos en la literatura, entonces el aut\'omata puede ser \'util en el seguimiento del desarrollo de un tumor en la vida real. También se pretende contribuir en la obtenci\'on de algoritmos eficientes para la representaci\'on gr\'afica en tres dimensiones del proceso de crecimiento de un tumor y el manejo de la gran cantidad de c\'elulas que intervienen en la simulaci\'on.
Se enumeran a continuaci\'on los objetivos espec\'ificos de este trabajo:
\begin{itemize}
    \item Definir una red de mundo pequeño mediante el modelo Watts-Strogatz.
    \item Definir el conjunto de c\'elulas y la funci\'on de vecindad del aut\'omata a partir de la red creada haciendo uso de conexiones entre las c\'elulas.
    \item Definir un conjunto de estados para las c\'elulas del aut\'omata que permitan representar las distintas entidades biol\'ogicas que se tienen en cuenta en el modelo, entre las cuales est\'an presentes las c\'elulas normales, cancer\'igenas y las inmunol\'ogicas.
    \item Definir una funci\'on de transición que, siguiendo ciertas reglas, permita describir: la evoluci\'on tumoral durante las etapas de su desarrollo, como afecta el tumor a los tejidos sanos, la interacci\'on del tumor con el sistema inmunol\'ogico, la migraci\'on de c\'elulas cancer\'igenas a trav\'es de los tejidos sanos y, finalmente, la met\'astasis.
    \item Comparar los resultados obtenidos con datos expetimentales encontrados en la literatura.
    \item Representar gr\'aficamente los procesos presentes en la simulaci\'on.
    \item Desarrollar un modelo din\'amico en cuanto a los par\'ametros y factores que influyen en la simulaci\'on para obtener resultados m\'as realistas.
    \item Implementar algoritmos eficientes para procesar grandes cantidades de células y sus conexiones.
\end{itemize}