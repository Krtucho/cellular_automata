\section*{S\'intesis}
La comprensi\'on del ciclo vital tumoral resulta de importancia crucial tanto para la investigaci\'on del c\'ancer como para la sanidad p\'ublica. La mayor parte de los esfuerzos en el campo de la modelaci\'on matem\'atica y computacional est\'an enfocados en estudiar el desarrollo del tumor durante las etapas tempranas, donde la mortalidad es muy baja. Como los comportamientos del tumor en las fases avanzadas de su desarrollo son las que presentan un peligro inminente para la vida del paciente, es necesario enfocar nuestro trabajo en producir representaciones matem\'aticas y herramientas computacionales que permitan estudiar dichos comportamientos. El presente modelo estoc\'astico de aut\'omatas celulares constituye un primer acercamiento a la reproducci\'on del crecimiento avascular y vascular del tumor, donde se muestran los comportamientos espec\'ificos de cada etapa, y basado en el proceso de acumulaci\'on de mutaciones. Adoptando como objeto de estudio el tipo de c\'ancer conocido como carcinoma y tomando en cuenta una variedad de interacciones y transiciones entre los diferentes tipos de c\'elulas cancer\'igenas y normales, se reproduce: el crecimiento tumoral hacia las distintas capas de tejidos, la invasi\'on del estroma del \'organo, el desplazamiento de las c\'elulas migratorias a trav\'es del tejido, la penetraci\'on y transporte a trav\'es del sistema circulatorio, la extravasaci\'on, la formaci\'on de nuevas micromet\'astasis y el per\'iodo de dormancia. Como representaci\'on de las localizaciones donde se desarrolla el c\'ancer se usa una red de mundo peque\~no generada a partir del modelo Watts-Strogatz, que se interpreta como el mapa de conexiones de las c\'elulas del tejido. Finalmente, el modelo permite obtener visualizaciones de todo el proceso y verificar c\'omo afecta al ciclo vital del c\'ancer la variaci\'on de sus distintos par\'ametros, emulando as\'i los efectos de posibles tratamientos.

\section*{Abstract}
The understanding of the tumor life cycle is of crucial importance for both cancer research and public health. Most of the efforts in the field of mathematical and computational modeling are focused on studying the development of the tumor during the early stages, where mortality is very low. Since the behavior of the tumor in the advanced stages of its development present an imminent danger to the life of the patient, it is necessary to focus our work on producing mathematical representations and computational tools to understand such behaviors. The present stochastic cellular automata model is an initial proposal of a model that reproduces the avascular and vascular growth of the tumor, where the behavior of each stage is based on the process of accumulation of tumor mutations. Adopting as an object of study the type of cancer known as carcinoma and considering a variety of interactions and transitions between the different types of normal and cancer cells, the model reproduces: the tumor growth towards the different layers of tissue, the invasion of the stroma of the organ, the displacement of the migratory cells through the tissue, the penetration and transport through the circulatory system, the extravasation, the formation of new micrometastasis and the dormancy period. As a representation of the locations where the tumor development takes place, a small world network generated from the Watts-Strogatz model is used. The network is interpreted as the connections map of the tissue cells. Finally, the model allows to obtain visualizations of the whole process and to verify how the variation of its different parameters affects the cancer life cycle, thus emulating the effects of possible treatments.

\section*{Opini\'on de los tutores}
El trabajo ``Aut\'omata celular estoc\'astico en redes complejas para el estudio de la invasi\'on, migraci\'on y met\'astasis del c\'ancer'' presentado por el estudiante Darien Viera Barredo constituye un resultado notable en el campo de la Biomec\'anica y Mec\'anica Computacional. La modelaci\'on matem\'atica, f\'isica y computacional de fen\'omenos biol\'ogicos es un desaf\'io extraordinario que requiere de la colaboraci\'on de los investigadores que pertenecen a estos campos. En la actualidad el c\'ancer es una de las principales causas de muerte en el mundo. Contar con una investigaci\'on que brinde la implementaci\'on de un grupo de algoritmos que en su conjunto proporcione una propuesta de soluci\'on al problema de la simulaci\'on computacional de la evoluci\'on de tumores tiene gran relevancia desde el punto de vista cient\'ifico en la rama de investigaci\'on de la Biomec\'anica Computacional.

Debido a la importancia del estudio computacional de la invasi\'on, migraci\'on y met\'astasis del c\'ancer es que surgi\'o la idea de crear una plataforma interactiva de la mec\'anica de materiales heterog\'eneos con las herramientas computacionales relacionada con esta l\'inea de desarrollo. Brindar a diferentes especialistas, tales como m\'edicos, radi\'ologos, onc\'ologos, f\'isicos, matem\'aticos, etc., una propuesta de estudio desde el punto de vista de la biomec\'anica computacional, sobre la evoluci\'on de la invasi\'on, migraci\'on y met\'astasis del c\'ancer a partir de la implementaci\'on de t\'ecnicas del aut\'omata celular resulta un desaf\'io para la comunidad cient\'ifica. 

Para la implementaci\'on de este proyecto de investigaci\'on, el diplomante debi\'o asimilar un volumen considerable de informaci\'on, de dis\'imiles campos, referente a muchos temas de la biomedicina y aspectos computacionales inherentes a ella. A partir de esta informaci\'on Darien, implement\'o un modelo de aut\'omatas celulares que presenta de forma integral las etapas avascular y vascular del desarrollo tumoral, as\'i como los procesos de invasi\'on, migraci\'on y met\'astasis a diferencia de otros modelos presentes en la literatura cient\'ifica que se concentran en representar parcialmente el ciclo de vida del c\'ancer. Basado en la bibliograf\'ia consultada se propone un conjunto de hip\'otesis que describen de forma idealizada el desarrollo tumoral tomando como base las marcas distintivas del c\'ancer y el proceso de acumulaci\'on de mutaciones de la c\'elula cancer\'igena, y se utilizan en la concepci\'on del modelo. El modelo sigue un riguroso proceso de concepci\'on, en el que se recogen numerosas definiciones y procedimientos que constituyen una fuerte base para trabajos futuros. Con respecto a la bibliograf\'ia consultada el modelo presenta varios par\'ametros que permiten regular los distintos procesos del desarrollo tumoral y los mecanismos de propagaci\'on descritos anteriormente. Desde un punto de vista biol\'ogico posee relevancia ya que se pueden probar los efectos de tratamientos potenciales mediante la variaci\'on de dichos par\'ametros. 

El trabajo escrito presenta una estructura clara y organizada que permite f\'acil comprensi\'on de los contenidos incluidos. Adem\'as de presentar los resultados obtenidos en su trabajo, el diplomante presenta elementos t\'ecnicos acerca de la programaci\'on como un grupo de t\'ecnicas y conocimientos del aut\'omata aplicados a la biomedicina, los cuales utiliza en el desarrollo de su trabajo, que hacen del documento escrito un buen material de referencia para los futuros trabajos relacionados con el tema.
El diplomante ha trabajado con gran dedicaci\'on durante toda su trayectoria, lo caracterizan su constancia y dedicaci\'on al trabajo de investigaci\'on. Debe destacarse que tuvo que asimilar en un per\'iodo muy corto todo un n\'umero de conceptos y definiciones referentes a otros campos de la computaci\'on aplicado a la biomedicina, aspecto de vital relevancia para el resultado obtenido y que no est\'an incluidos en el plan de estudio de la especialidad. 
