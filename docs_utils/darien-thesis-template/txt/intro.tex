\chapter{Introducci\'on} 
\label{sec-intro}
La modelaci\'on matem\'atica, f\'isica y computacional de fen\'omenos bi\'ologicos es un desaf\'io extraordinario que requiere de la colaboraci\'on de los investigadores que pertenecen a estos campos. Esta cooperaci\'on hace posible que el m\'etodo experimental tradicionalmente utilizado en las Ciencias Biol\'ogicas sea complementado con la modelaci\'on matem\'atica, que a su vez puede ser utilizada como una herramienta para generar y probar hip\'otesis, contribuyendo en la direcci\'on de investigaciones experimentales mientras que los resultados de estos experimentos pueden ayudar con el refinamiento del modelo. El objetivo de este tipo de investigaci\'on es que mediante la eventual iteraci\'on entre teor\'ia y la experimentaci\'on se logre un entendimiento conceptual m\'as profundo de c\'omo interact\'uan los procesos biol\'ogicos. Adem\'as los modelos matem\'aticos pueden ser usados para ayudar en la concepci\'on y dise\~no de estrategias terap\'euticas~\cite{bellomo}.

En la actualidad el c\'ancer es una de las principales causas de muerte en el mundo. En las naciones desarrolladas constituye un problema fundamental ya que en los \'ultimos a\~nos se ha movido de la s\'eptima posici\'on a la segunda en la lista de enfermedades fatales. Seg\'un el Instituto Nacional del C\'ancer de Estados Unidos (NCI) en el $2012$ se diagnosticaron $14.1$ millones de nuevos casos y ocurrieron $8$.$2$ millones de muertes relacionadas con la enfermedad a nivel mundial. Se espera que el n\'umero de casos anuales aumente hasta llegar a $23$.$6$ millones. En particular, en ese a\~no el c\'ancer ocup\'o el primer lugar en la lista de las causas de defunciones en Cuba seg\'un la Direcci\'on Nacional de Registros M\'edicos y Estad\'isticas de la Salud~\cite{ariel}. Los cuidados, tratamientos, investigaciones, ensayos cl\'inicos y otras diversas labores relacionadas con la enfermedad conllevan un gasto econ\'omico considerable. Solo en los Estados Unidos se estima que durante el a\~no $2017$ se destinaron $147.3$ mil millones de d\'olares al cuidado y tratamiento de pacientes. Por estos motivos, la lucha contra el c\'ancer es de gran importancia para la sanidad p\'ublica y la econom\'ia alrededor del mundo.

El c\'ancer es un conjunto de enfermedades que afectan a una gran variedad de seres vivos y que tienen como caracter\'istica fundamental la presencia de un grupo de c\'elulas anormales que crece sin control, obviando las normas de la divisi\'on celular~\cite{robins}. Afecta de forma especial al ser humano donde su aparici\'on y desarrollo constituye un peligro para la vida. La malignidad del c\'ancer es variable y depende de la velocidad de crecimiento de las c\'elulas cancer\'igenas, la capacidad de estas \'ultimas de propagarse a otros tejidos y la posibilidad de reaparecer una vez que son removidas quir\'urgicamente. La existencia de estos tumores malignos tiene repercusiones graves, puesto que consume los nutrientes necesarios para el funcionamiento del organismo, causan hemorragias e infecciones cuando se ulceran a trav\'es de superficies naturales adyacentes y en casos extremos, el tejido tumoral puede reemplazar en su totalidad al tejido sano, eliminando la funcionalidad del \'organo invadido~\cite{robins}.

Las c\'elulas cancer\'igenas muestran pronunciadas diferencias morfol\'ogicas con las c\'elulas normales: no presentan la compartimentaci\'on y estructura interna de una c\'elula sana, de forma que pierden sus funciones y enfocan su metabolismo enteramente a la proliferaci\'on descontrolada~\cite{oportunism}. Estos cambios hacen que sea dif\'icil creer que se comporten como masas de c\'elulas desorganizadas, difusas y aleatorias, en cambio sugieren que son sistemas emergentes oportunistas~\cite{oportunism}. Si esta hip\'otesis es verdadera, las investigaciones deben enfocarse en el tumor macrosc\'opico pero tomando en cuenta las mec\'anicas celulares, trat\'andolo como un sistema complejo, din\'amico y autoorganizado~\cite{kansal}. Debido a la naturaleza compleja de los sistemas subyacentes en el comportamiento de los tumores malignos y al limitado entendimiento de los mecanismos del crecimiento tumoral, el desarrollo de modelos matem\'aticos o computacionales realistas es una tarea dif\'icil. En trabajos sobre el crecimiento de tejidos se han utilizado enfoques que se centran en representar e investigar las interacciones entre las c\'elulas. En especial, los aut\'omatas celulares son uno de los modelos m\'as exitosos, lo cual se traduce en una gran cantidad de investigaciones. Es com\'un que los sistemas estudiados mediante aut\'omatas celulares muestren comportamientos emergentes y autoorganizaci\'on, caracter\'isticas presentes en el c\'ancer, haciendo de este modelo matem\'atico una elecci\'on natural para estudiar esta compleja enfermedad.

\section{Antecedentes}
El t\'ermino crecimiento se utiliza tanto para indicar un incremento en el tama\~no o volumen como para expresar un aumento en la poblaci\'on. En la historia de la modelaci\'on de procesos de crecimiento existen tres investigadores cuyos esquemas b\'asicos todav\'ia se consideran adecuados para el tratamiento de numerosos problemas pr\'acticos: el modelo de crecimiento exponencial de Thomas R. Malthus~($1798$), la ley de crecimiento de Benjamin Gompertz~($1825$) y el modelo de crecimiento log\'istico de Pierre F. Verhulst~($1838$). Estos modelos expresan el comportamiento de la poblaci\'on en funci\'on del tiempo dado un ritmo de crecimiento, donde la poblaci\'on y el tiempo son variables que poseen valores continuos. En~\cite[p\'aginas $129$-$131$]{book} se presenta una revisi\'on del planteamiento y las diferencias entre estos modelos cl\'asicos. Se ha observado que numerosos fen\'omenos biol\'ogicos se comportan de acuerdo a dichos modelos bajo la influencia de distintos factores y condiciones del entorno, sin embargo, las caracterizaciones m\'as adecuadas provienen de los modelos de Verhulst y Gompertz. En~\cite{gompertz} se expone un an\'alisis detallado de las relaciones entre los modelos derivados de estos \'ultimos que se han utilizado para representar el crecimiento tumoral, tomando como base de la comparaci\'on los datos provenientes de sistemas de cultivo de esferoides tumorales multicelulares~(de ahora en adelante MTS, \emph{multicellular tumor spheroids}). Este trabajo presenta, adem\'as, un ejemplo simple del proceso de concepci\'on de un modelo de este tipo.

En el \'area de la modelaci\'on de variables continuas, la mec\'anica de medios continuos~(de ahora en adelante MMC) provee otro enfoque para tratar con el crecimiento de un tumor. En este contexto, el comportamiento de un material se rige por medio de ecuaciones constitutivas que caracterizan las propiedades de dicho material, y leyes de balance entre las que se encuentran la ley de balance de la masa y de la energ\'ia. De esta forma, la evoluci\'on del tumor se deriva de las ecuaciones de balance, as\'i como de principios de conservaci\'on suplementados con leyes de difusi\'on para describir la evoluci\'on de los nutrientes que el tumor recibe para su desarrollo~\cite{ariel}. Algunos de los trabajos de estudio de tumores basados en la MMC~(ver por ejemplo~\cite{preziosi} y~\cite{preziosi2}) consideran que todas las c\'elulas del organismo poseen una tensi\'on ideal y si en alg\'un momento esta tensi\'on var\'ia existen mecanismos para devolverla a su estado ideal. Esta hip\'otesis tiene su base en los mecanismos de homeostasis del organismo, que consiste en la capacidad del mismo para mantener una condici\'on interna estable. Por tanto, la aparici\'on de un tumor se interpreta como una falla en los mecanismos que hacen que la c\'elula, y por extensi\'on el tejido, recuperen la tensi\'on ideal. Se puede mencionar como un ejemplo de este an\'alisis el modelo presentado en~\cite{fernando2}, en el cual se estudian las interacciones entre el crecimiento anisotr\'opico y el estado de tensi\'on de un tumor avascular rodeado por un medio externo. La construcci\'on y resoluci\'on num\'erica de modelos matem\'aticos de crecimiento de tumores sujetos a diferentes geometr\'ias, estructuras internas y condiciones de frontera en el marco de la mec\'anica de medios continuos se presenta en~\cite{ariel2,ariel3}.

Los enfoques anteriores constituyen los m\'etodos tradicionales para modelar el crecimiento tumoral, pero en los \'ultimos a\~nos numerosos investigadores han explotado las facilidades que ofrecen los aut\'omatas celulares para razonar en t\'erminos de individuos, dado que en este marco la poblaci\'on y el tiempo son variables discretas. La din\'amica de un aut\'omata celular depende de un conjunto de estados, que toman las c\'elulas de la poblaci\'on, y una funci\'on de transici\'on, que expresa los cambios que ocurren entre dichos estados. No obstante, resulta complejo representar la din\'amica del crecimiento tumoral desde las interacciones entre las c\'elulas que lo conforman, por lo que usualmente se lleva a cabo un proceso de inferencia de la funci\'on de transici\'on a partir de alg\'un modelo continuo que describa dicho crecimiento de forma macrosc\'opica. En~\cite{guinot} se presenta una metodolog\'ia de este proceso de inferencia a partir de modelos de ecuaciones diferenciales ordinarias y un ejemplo de su aplicaci\'on se muestra en~\cite{ruben} donde se obtiene la funci\'on de transici\'on para un aut\'omata celular estoc\'astico a partir del modelo desarrollado en~\cite{fernando}.

Los aut\'omatas celulares han sido utilizados extensamente para simular el crecimiento de un tumor durante la etapa avascular~\cite{dormann,kansal2}, la invasi\'on celular~\cite{anderson} y sus interacciones con varios factores del entorno~\cite{rejniak}. En~\cite{kansal} se desarrolla un modelo de aut\'omata celular en tres dimensiones para representar el crecimiento de un tumor maligno en el cerebro conocido como glioblastoma multiforme(GBM), en el que utilizan una triangulaci\'on de Delaunay como espacio celular y reproducen el modelo de Gompertz de forma muy precisa. Este trabajo es ampliado posteriormente en~\cite{kansal3} donde incorporan las interacciones entre el tumor y el entorno donde se desarrolla, obteni\'endose comportamientos emergentes en la evoluci\'on del aut\'omata que imitan la invasi\'on y migraci\'on de las c\'elulas cancer\'igenas en cultivos de MTS. En~\cite{dormann}, se simula el crecimiento de un tumor en etapa avascular usando un aut\'omata celular h\'ibrido, a partir de datos recogidos mediante la experimentaci\'on con MTS exhibiendo la formaci\'on auto organizada de una estructura dividida en capas en el interior del tumor. Un an\'alisis cuantitativo del crecimiento de una subpoblaci\'on de c\'elulas cancer\'igenas dentro de un tumor previamente homog\'eneo se muestra en~\cite{kansal2}. Finalmente, una comparaci\'on de las fortalezas y debilidades de varios modelos basados en c\'elulas, incluidos los aut\'omatas celulares, se presenta en~\cite{rejniak}.

\section{Motivaci\'on}
En los antecedentes citados anteriormente se observa que los aut\'omatas celulares se han utilizado extensamente para modelar el desarrollo de tumores malignos, principalmente durante la etapa avascular. En esta etapa el tumor presenta el menor riesgo para la salud, ya que se encuentra en los primeros momentos de su desarrollo y posee una malignidad limitada. A medida que avanza el desarrollo del tumor aumenta su malignidad, ganando el potencial de penetrar en distintos tejidos, invadir el \'organo donde se origin\'o y migrar a nuevas localizaciones. Pero estos comportamientos solo toman lugar en un tumor que posee el desarrollo suficiente para llevar a cabo su vascularizaci\'on, es decir, inducir el crecimiento de numerosos capilares sangu\'ineos en su interior. Por tanto, la mayor parte de los modelos que representan el desarrollo de un tumor maligno no poseen entre sus objetivos reproducir los mecanismos responsables de su propagaci\'on. Dado que la principal causa de fallecimiento en los pacientes con c\'ancer son las complicaciones que surgen a ra\'iz de su diseminaci\'on en el organismo, resulta necesario desarrollar un modelo predictivo que tome en cuenta los factores y mec\'anicas celulares que intervienen en este proceso. El modelo presentado en este trabajo surge como una respuesta a esta necesidad. 

La elecci\'on de los aut\'omatas celulares como marco de trabajo se debe al conjunto de ventajas que poseen frente a los enfoques tradicionales de modelos que utilizan ecuaciones diferenciales de variables continuas~\cite{guinot}. No est\'an sujetos a la inestabilidad en el sentido cl\'asico de la palabra ya que el n\'umero de estados que toman los elementos de la poblaci\'on es finito. Como se expuso anteriormente resultan m\'as adecuados para el razonamiento en t\'erminos de individuos porque las variables de estado son discretas. Por \'ultimo, poseen la capacidad de modelar patrones de comportamiento complejos a partir de un conjunto simple de reglas de transici\'on. No obstante, se demostr\'o en~\cite{ruben} que pueden ser utilizados como alternativa para obtener la evoluci\'on de las variables de un modelo continuo en una forma interactiva y m\'as cercana a la realidad. 

Con el objetivo de reproducir los mecanismos de propagaci\'on del c\'ancer es indispensable tener en cuenta las interacciones entre las c\'elulas cancer\'igenas y el sistema circulatorio. La concepci\'on del aut\'omata celular se apoya en la hip\'otesis de que un tejido vivo puede ser modelado a partir de redes complejas~\cite{complexnetworks}, en particular las de mundo peque\~no. Por la naturaleza de su construcci\'on estas redes poseen dos tipos de conexiones: entre c\'elulas cercanas y entre c\'elulas distantes. Desde el punto de vista biol\'ogico las conexiones entre c\'elulas cercanas indica la posibilidad de interactuar dada su cercan\'ia f\'isica, y las conexiones entre c\'elulas distantes indica la posibilidad de interactuar dado su v\'inculo a trav\'es sistema circulatorio. Las redes de mundo peque\~no son capaces de representar de forma pr\'actica y conveniente las interacciones entre las c\'elulas cancer\'igenas y el sistema circulatorio.

\section{Objetivos y contribuciones}
El objetivo general del presente trabajo consiste en modelar la invasi\'on, migraci\'on y met\'astasis, mecanismos fundamentales que el c\'ancer utiliza para propagarse, empleando los aut\'omatas celulares definidos en redes complejas de mundo peque\~no como marco de trabajo. Como objetivo secundario tiene el de crear un marco de trabajo que funcione como base de futuras investigaciones, que recoja todas las etapas y procesos que forman parte del desarrollo del c\'ancer e incluya la utilizaci\'on de t\'ecnicas novedosas relacionadas con la modelaci\'on mediante aut\'omatas celulares.

Los objetivos espec\'ificos de este trabajo se enumeran a continuaci\'on:

\begin{itemize}
\item [I.] Definir el conjunto de c\'elulas y la funci\'on de vecindad del aut\'omata a partir de una red compleja de mundo peque\~no, que se genera mediante el modelo Watts-Strogatz.
\item [II.] Establecer un conjunto de estados para las c\'elulas del aut\'omata que permita representar las distintas entidades biol\'ogicas que se toman en cuenta en el modelo, entre las que se encuentran distintos tipos de c\'elulas normales y cancer\'igenas.
\item [III.] Definir una funci\'on de transici\'on que, mediante distintas reglas, describa: la evoluci\'on tumoral durante ambas etapas de su desarrollo, las interacciones entre el tumor y los tejidos sanos circundantes que conllevan a su invasi\'on, la migraci\'on de c\'elulas cancer\'igenas a trav\'es de los tejidos sanos y, finalmente, la met\'astasis. Esta \'ultima regla debe utilizar plenamente la red de mundo peque\~no para su implementaci\'on.
\item [IV.] Comparar los resultados obtenidos con datos y evidencias experimentales existentes en la literatura.
\item [V.] Comprobar la capacidad predictiva del modelo ante distintos valores del conjunto de par\'ametros.
\item [VI.] Obtener visualizaciones de los todos los procesos que se manifiestan en el aut\'omata celular y que se especifican en la funci\'on de transici\'on.
\end{itemize}

Los aspectos novedosos presentes en este trabajo se enumeran a continuaci\'on:

\begin{itemize}
\item [I.] Se concibe un modelo de aut\'omatas celulares que presenta de forma integral las etapas avascular y vascular del desarrollo tumoral, as\'i como los procesos de invasi\'on, migraci\'on y met\'astasis a diferencia de otros modelos presentes en la literatura cient\'ifica que se concentran en representar parcialmente el ciclo de vida del c\'ancer. 
\item [II.] Se propone un conjunto de hip\'otesis que describen de forma idealizada el desarrollo tumoral tomando como base las marcas distintivas del c\'ancer y el proceso de acumulaci\'on de mutaciones de la c\'elula cancer\'igena, las cuales se utilizan en la concepci\'on del modelo. 
\item [III.] Se definen varios par\'ametros que permiten regular los distintos procesos del desarrollo tumoral y los mecanismos de propagaci\'on descritos anteriormente. Desde un punto de vista biol\'ogico esto es relevante pues se pueden probar los efectos de tratamientos potenciales mediante la variaci\'on de dichos par\'ametros.
\end{itemize}