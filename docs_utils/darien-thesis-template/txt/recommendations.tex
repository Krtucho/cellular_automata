\chapter{Recomendaciones}
Es pertinente a esta secci\'on proponer l\'ineas de investigaci\'on cuyos objetivos permitan complementar y validar tareas como b\'usqueda de par\'ametros, selecci\'on de las representaciones m\'as adecuadas, exploraci\'on de otros modelos discretos que posean otras concepciones como las simulaciones basadas en agentes, entre otras. Las recomendaciones dirigidas a ampliar el modelo se presentan a continuaci\'on:

\begin{itemize}
\item En primer lugar, implementar el modelo en tres dimensiones y evaluar la posibilidad de su ejecuci\'on en un sistema computacional de altas prestaciones. 

\item Acoplar utilizando la metodolog\'ia propuesta en~\cite{guinot} otros modelos continuos que describan los comportamientos agresivos del c\'ancer utilizando otros enfoques~(e.g. \cite{preziosi,preziosi2,kansal3,vascular}) y comparar los resultados obtenidos de estos modelos derivados.

\item Explorar en los modelos derivados se\~nalados en el punto anterior la implementaci\'on de las reglas alternativas de la aparici\'on e intravasaci\'on directa de c\'elulas migratorias que se expusieron en la secci\'on de conclusiones.

\item Ampliar el modelo para que incorpore mecanismos que permitan reproducir el recrecimiento tumoral cuando se lleva a cabo una resecci\'on quir\'urgica o el efecto de terapias dirigidas a disminuir el tama\~no del tumor para su estudio. 

\item En el sentido del punto anterior, se deber\'ia incluir las interacciones de las c\'elulas cancer\'igenas con el sistema inmunitario de forma m\'as directa consider\'andolas entidades individuales dentro del aut\'omata con uno o varios estados propios~\cite{ruanxiaoca}.

\item Proponer la realizaci\'on de un estudio para concebir otras representaciones de un tejido vivo que puedan ser utilizadas en los modelos derivados~(e.g. teselaciones de Voronoi como se utilizan en~\cite{kansal,kansal2,kansal3}, redes libre de escala como una alternativa a las redes de mundo peque\~no). 

\item La utilizaci\'on de los gradientes de la concentraci\'on de nutrientes puede ser ampliada en trabajos posteriores mediante la representaci\'on de condiciones homog\'eneas y heterog\'eneas en el interior del estroma, como por ejemplo que dicho gradiente aumenta a medida que nos acercamos a una conexi\'on distante provocando que la migraci\'on y el crecimiento tumoral se vea sesgado a ocurrir hacia esa celda~\cite{anderson,kansal3,vascular,chaplain}.

\item En el presente modelo todas las neoplasias de una misma simulaci\'on poseen los mismos par\'ametros por lo que presentan comportamientos similares. Como una ampliaci\'on se puede proponer desarrollar un modelo que permita obtener crecimientos, migraciones y capacidades metast\'asicas variables para cada neoplasia. 

\item Concebir un modelo de aut\'omatas celulares que permita reproducir el crecimiento de vasos sangu\'ineos en un tejido en base a las concentraciones de factores de crecimiento angiog\'enicos. Tiene como objetivo reproducir de forma m\'as realista la angiog\'enesis tumoral~\cite{book,vascular,angiogenesis}.
\end{itemize} 