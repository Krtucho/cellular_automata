\section*{Agradecimientos}

\begin{flushright}
\textit{``Errare humanum est,\\ sed perseverare diabolicum.''}
\end{flushright}

Creo en la idea de que una persona es la sumatoria de todas las experiencias que ha tenido en su vida. Estas experiencias vienen por etapas donde algunas son m\'as intensas y otras m\'as calmadas. No s\'e qu\'e me espera a continuaci\'on, pero s\'i estoy seguro de que las experiencias que he tenido en mis a\~nos de estudio en la Facultad de Matem\'atica y Computaci\'on de la Universidad de la Habana son incomparables gracias a los excepcionales individuos con los que he tenido la suerte de interactuar. 

Estar\'e eternamente agradecido con mis tutores Reinaldo Rodr\'iguez Ramos y Ruben Interian quienes me introdujeron a este mundo dentro de la ciencia al que he llegado a amar. Quiero hacer una menci\'on especial a los colaboradores Ariel Ram\'irez Torres y Roc\'io Rodr\'iguez S\'anchez que se brindaron a ayudar cuando m\'as lo necesit\'abamos. Agradezco al colectivo completo de la facultad, en especial a todos los profesores que en alg\'un momento me tuvieron como disc\'ipulo y con los que compart\'i mis logros y mis fracasos. 

En el \'ambito personal tengo que agradecer a todas las amistades que he hecho, a mis compa\~neros de beca y de aula, a los cibern\'eticos, f\'isicos, matem\'aticos, qu\'imicos, bi\'ologos y comunicadores que pertenecen a mi c\'iculo cercano, porque comparten conmigo la misma pasi\'on y por haber luchado como aliados o enemigos tantas veces en defensa del trono. Decir nombres har\'ia de estos agradecimientos interminables pero advierto que los recuerdo a todos y cada uno de ustedes. Nunca los olvidar\'e.

Finalmente pero no por ello menos importante, y aunque haya estado con ellos desde que nac\'i constituyen las personas m\'as excepcionales que conozco: mi familia. Estar\'e por siempre agradecido a mi madre por quererme tanto, por ser un ejemplo de fuerza y voluntad y por aguantarme todos estos a\~nos; a mi abuela por ser la mujer m\'as cari\~nosa del mundo cuando est\'a contenta, la mejor cocinera cuando est\'a en la cocina y la peleadora m\'as despiadada cuando est\'a enfadada; a mi abuelo que aunque haya dejado un gran vac\'io en nuestros corazones siempre ser\'a un ejemplo de seriedad, dignidad y honradez; a mi hermano por ser mi contrincante y quien me ha hecho esforzarme para no quedarme detr\'as; y a mi padre por ser un ejemplo de todo lo que no puedo ser en esta vida.